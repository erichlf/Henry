Henry uses a Fourier series to decompose the non-linear differential equations into a sum of
orthogonal basis functions. Then Henry uses what he calls Galerkin's Method to solve for the Fourier
coefficients. In this case Galerkin's Method refers to substituting the Fourier series back into the
differential equations, multiplying both differential equations by orthogonal functions, and then
integrating over the rectangular domain. First, new variables were introduced to further simplify
the boundary conditions. Letting

\begin{equation} \label{3.1)} 
    \Psi = \psi '-y'\csp C = c'-\frac{x'}{\xi } 
\end{equation}

and after dropping the primes, and using subscripts to denote differentiation,
the problem in terms of $\Psi \spbox{and} C$ becomes

\begin{equation} \label{ZEqnNum626661} 
    a\nabla ^{2} \Psi =C_{x} +\frac{1}{\xi } 
\end{equation} 
\begin{equation} \label{ZEqnNum183863} 
    b\nabla ^{2} C=\Psi _{y} C_{x} -\Psi _{x} C_{y} +\frac{1}{\xi } \Psi _{y} +C_{x} +\frac{1}{\xi } 
\end{equation} 
\begin{equation} \label{ZEqnNum529054} 
    C_{y} =0 \csp \Psi =0, \spbox{at} y = 0,1; \quad C = 0\csp \Psi
    _{x} = 0, \spbox{at}x=0,\xi
\end{equation}

$\Psi\spbox{and} C$ can be represented by a double Fourier series that satisfies
\eqref{ZEqnNum529054} \cite{Henry60}

\begin{equation} \label{ZEqnNum807044} \Psi =\sum _{m=1}^{\infty } \sum
_{n=0}^{\infty }A_{m,n} \sin \left(m\pi y\right)\cos \left(n\pi \frac{x}{\xi }
\right) \end{equation}

\begin{equation} \label{ZEqnNum170840} C=\sum _{r=0}^{\infty } \sum
_{s=1}^{\infty }B_{r,s} \cos \left(r\pi y\right)\sin \left(s\pi \frac{x}{\xi }
\right) \end{equation}

Substituting equations \eqref{ZEqnNum807044} and \eqref{ZEqnNum170840} into
\eqref{ZEqnNum626661} and \eqref{ZEqnNum183863}, then multiplying
\eqref{ZEqnNum626661} by $4\sin \left(g\pi y\right)\cos \left(h\pi \frac{x}{\xi
} \right)$, and multiplying \eqref{ZEqnNum183863} by $4\cos \left(g\pi
y\right)\sin \left(h\pi \frac{x}{\xi } \right)$ integrating over the rectangular
domain, results in a few different integrals to be evaluated. But, before
evaluating any integral one should first realize two things; first,

\begin{equation} \label{3.7)} \sin (g\pi )=0 \quad \forall g\in {\mathbb Z}
\end{equation}
and second, 
\begin{equation} \label{ZEqnNum821421} \cos \left(h\pi \right)\equiv
\left(-1\right)^{h} \quad \forall h\in {\mathbb Z} \end{equation}

These two facts will come in very handy when simplifying the result of each
integral. 

For simplicity it will be easier to evaluate each integral separately, rather
than to treat the entire equation all at once. The left hand side of equation
\eqref{ZEqnNum626661} becomes

\begin{equation} \label{ZEqnNum923088} \begin{split} 
    a \pi ^{2} \sum _{m=1}^{\infty} \sum_{n=0}^{\infty} 
    & \left(m^{2} + \frac{n^{2}}{\xi ^{2}} \right) A_{m,n} \cdot \\
    & \int _{0}^{1} \int _{0}^{\xi} 4 \sin \left(g \pi y \right) 
    \sin \left(m\pi y\right) \cos \left(n \pi \frac{x}{\xi} \right)
    \cos \left(h \pi \frac{x}{\xi} \right) dxdy  
\end{split} \end{equation}

The right hand side of equation \eqref{ZEqnNum626661} will be split into two
parts, $C_{x} \spbox{and} \frac{1}{\xi } $, in which one obtains
\begin{equation} \label{ZEqnNum851626} \begin{split}
    \frac{\pi}{\xi} \sum _{r=0}^{\infty} \sum _{s=1}^{\infty}  
    & B_{r,s} \cdot s \cdot \\
    & \int_{0}^{1} \int _{0}^{\xi } 4 \sin \left(g \pi y\right) 
    \cos (r\pi y)\cos \left(h\pi \frac{x}{\xi} \right)\cos
    \left(s\pi \frac{x}{\xi} \right) dxdy  
\end{split} \end{equation}
and
\begin{equation} \label{ZEqnNum379567} \frac{1}{\xi } \int _{0}^{1} \int
_{0}^{\xi }4\sin \left(g\pi y\right)\cos \left(h\pi \frac{x}{\xi } \right)dxdy
\end{equation}
respectively.

Beginning with the simplest integral first, which is of course
\eqref{ZEqnNum379567}. First, notice that if $h\ne 0$ then the entire integral
will be zero. Integral \eqref{ZEqnNum379567} when evaluated using the above
observation becomes
\begin{equation*}
    \frac{1}{\xi } \int _{0}^{1} \int _{0}^{\xi }4\sin \left(g\pi y\right)dxdy
    =\frac{4}{\pi } \frac{1-\cos \left(g\pi \right)}{g} 
\end{equation*}

Using relationship \eqref{ZEqnNum821421} one obtains the result

\begin{equation*}
    \frac{1}{\xi } \int _{0}^{1} \int _{0}^{\xi }4\sin \left(g\pi y\right)dxdy
    \equiv \frac{1-\left(-1\right)^{g} }{g} 
\end{equation*}

and using the observation, if $g = 0$ then the entire integral will evaluate to
zero, and therefore integral \eqref{ZEqnNum379567} becomes

\begin{equation} \label{ZEqnNum309765} 
    \frac{1}{\xi } \int _{0}^{1} \int _{0}^{\xi } 4 \sin \left(g\pi y\right) 
    \cos \left(h\pi \frac{x}{\xi } \right)dxdy
    =\frac{4}{\pi } W\left(g,h\right) 
\end{equation}

where

\begin{equation}\label{3.13} 
    \ctfunc{W\left(g,h\right)}
    {0}{\text{for } h \ne 0}
    {0}{\text{for } g = 0}
    {\frac{1 - \left(-1 \right)^{g}}{g}}{\text{for } h = 0}. 
\end{equation}

When finishing the right hand side of equation \eqref{ZEqnNum626661} one should
notice the integral \eqref{ZEqnNum851626} will only be non-zero when $s=h$. This
results in \eqref{ZEqnNum851626} becoming

\begin{equation} \label{ZEqnNum929872} 
    \frac{\pi }{\xi } \sum _{r=0}^{\infty} B_{r,h} \cdot 
    h\int _{0}^{1} \int _{0}^{\xi} 4\sin \left(g\pi y\right)
    \cos (r \pi y)\cos ^{2} \left(h\pi \frac{x}{\xi } \right) dxdy 
\end{equation}

Using the double angle formula $\cos ^{2} x=\frac{\cos \left(2x\right)+1}{2} $,
and substituting this into \eqref{ZEqnNum929872}, one obtains

\begin{equation} \label{ZEqnNum253937} 
    \frac{\pi }{\xi } \sum _{r=0}^{\infty} B_{r,h} \cdot 
    h \int _{0}^{1}2\sin \left(g\pi y\right)\cos (r\pi y) 
    \int_{0}^{\xi } \left(\cos \left(2h\pi \frac{x}{\xi } \right) + 1 \right) dxdy
\end{equation}

Simplifying the integral further one uses the relation $2\sin \left(x\right)\cos
\left(y\right)=\sin \left(x+y\right)+\sin \left(x-y\right)$,
\eqref{ZEqnNum253937} becomes

\begin{equation} \label{ZEqnNum322465} \begin{split}
    \frac{\pi }{\xi } \sum _{r=0}^{\infty} B_{r,h} \cdot 
    & h\int _{0}^{1} \left( \sin \left[ \left( g + r \right) \pi y \right] + 
    \sin \left[ \left(g - r\right) \pi y \right] \right) \cdot \\
    & \int _{0}^{\xi} \left(\cos \left(2h\pi \frac{x}{\xi} \right) + 1\right) dxdy 
\end{split} \end{equation} 

To evaluate this integral, first look at the integral $\int _{0}^{\xi} 
\left(\cos \left(2h\pi \frac{x}{\xi } \right)+1\right) dx$. Evaluating this
integral one obtains

\begin{equation} \label{ZEqnNum494103} \int _{0}^{\xi } \left(\cos \left(2h\pi
\frac{x}{\xi } \right)+1\right) dx=\xi \end{equation}

Substituting \eqref{ZEqnNum494103} into \eqref{ZEqnNum322465} one obtains

\begin{equation} \label{ZEqnNum504644} \pi \sum _{r=0}^{\infty }B_{r,h} \cdot
h\int _{0}^{1} \left(\sin \left[\left(g+r\right)\pi y\right]+\sin
\left[\left(g-r\right)\pi y\right]\right) dy \end{equation}

Evaluating the integral $\pi \int _{0}^{1} \left(\sin \left[\left(g+r\right)\pi
y\right]+\sin \left[\left(g-r\right)\pi y\right]\right) dy$ results in

\begin{equation*} \begin{split}
    \pi \int _{0}^{1} \sin \left[\left(g + r\right)\pi y\right]
    & + \sin \left[\left(g - r\right) \pi y\right] dy \\
    & = \frac{\cos \left[\left(g + r\right)\pi \right] - 1}{g + r} 
    + \frac{\cos \left[\left(g - r \right) \pi \right] - 1}{g - r} 
\end{split} \end{equation*} 

Using the relationship \eqref{ZEqnNum821421} one obtains

\begin{equation*}
    \pi \int _{0}^{1} \left(\sin \left[\left(g+r\right)\pi y\right]+\sin
    \left[\left(g-r\right)\pi y\right]\right) dy\equiv \frac{\left(-1\right)^{g+r}
    -1}{g+r} +\frac{\left(-1\right)^{g-r} -1}{g-r} 
\end{equation*}

One should notice that this becomes undefined if $r=g$, and so one must evaluate
the integral for when $r=g$. When $r=g$ the integral becomes

\begin{equation*}
    \pi \int _{0}^{1} \sin \left(2g\pi y\right) dy = 0
\end{equation*} 

 Therefore the integral \eqref{ZEqnNum504644} becomes

\begin{equation*}
    \pi \sum _{r=0}^{\infty }B_{r,h} \cdot h\int _{0}^{1} \left(\sin
\left[\left(g+r\right)\pi y\right]+\sin \left[\left(g-r\right)\pi
y\right]\right) dy =\sum _{r=0}^{\infty }B_{r,h} \cdot h\cdot N\left(g,r\right)
\end{equation*}

where

\begin{equation} \label{3.19)} 
    N\left(g,r\right) = \begin{cases}
    0 & \text{for } r = g \\
    \frac{\left(-1\right)^{g + r} - 1}{g + r} + \frac{\left(-1\right)^{g - r} - 1}{g - r} 
    & \text{otherwise}\end{cases} 
\end{equation}

 To finish up with \eqref{ZEqnNum626661} integral \eqref{ZEqnNum923088} must be
 evaluated. One should notice that this integral is non-zero only when $m = g
 \spbox{and} n = h$, and therefore \eqref{ZEqnNum923088} becomes

\begin{equation} \label{ZEqnNum599031} 
    a \pi ^{2} \left(g^{2} + \frac{h^{2}}{\xi^{2}} \right)
    A_{g,h} \int _{0}^{1} \int _{0}^{\xi} 4 \sin ^{2} 
    \left(g \pi y\right) \cos ^{2} \left(h \pi \frac{x}{\xi} \right)dxdy 
\end{equation}

Using the double angle formulas, $\cos ^{2} x = \frac{\cos \left(2x\right) + 1}{2} $
and $\sin ^{2} x = \frac{1 - \cos \left(2 x \right)}{2} $, the integral
\eqref{ZEqnNum599031} becomes

\begin{equation} \label{3.21)} 
    a\pi ^{2} \left(g^{2} +\frac{h^{2} }{\xi ^{2} } \right)
    A_{g,h} \int _{0}^{1} \left(1-\cos \left(2g\pi y\right)\right)
    \int_{0}^{\xi } \left(\cos \left(2h \pi \frac{x}{\xi } \right) + 1\right) dxdy
\end{equation}

Since $h$ can be zero, one must evaluate this integral for the two cases
$h = 0 \spbox{and} h \ne 0$, resulting in

\begin{equation} \label{3.22)} 
    \begin{array}{l} {a\pi ^{2} \left(g^{2}
    +\frac{h^{2} }{\xi ^{2} } \right)A_{g,h} \int _{0}^{1} \left(1-\cos \left(2g\pi
    y\right)\right)\int _{0}^{\xi } \left(\cos \left(2h\pi \frac{x}{\xi }
    \right)+1\right)dxdy } \\ {=\varepsilon \left(h\right)\cdot a\pi ^{2}
    \left(g^{2} +\frac{h^{2} }{\xi ^{2} } \right)A_{g,h} \cdot \xi } \end{array}
\end{equation}

where

\begin{equation} \label{3.23)} 
    \varepsilon \left(h\right) = \begin{cases}
        2 & h = 0 \\ 
        1 & \text{otherwise}\end{cases} 
\end{equation} 

The more complex equation \eqref{ZEqnNum183863} will have to be dealt with in
several parts. First, the left hand side of equation \eqref{ZEqnNum183863}
becomes

\begin{equation} \label{ZEqnNum189325} \begin{split} 
    b\pi ^{2} \sum _{r=0}^{\infty } \sum_{s=1}^{\infty } 
    & \left(r^{2} + \frac{s^{2} }{\xi^{2} } \right) B_{r,s} \\ 
    & \int_{0}^{1} \int_{0}^{\xi } 4 \cos \left(g \pi y\right) 
    \cos \left(r\pi y\right) \sin\left(s \pi \frac{x}{\xi} \right)
    \sin \left(h \pi \frac{x}{\xi} \right) dxdy
\end{split} \end{equation}

 To deal with the right hand side of the equation it will be much simpler to
 split it into its separate derivatives. However, grouping the derivatives $\Psi
 _{y} C_{x} \spbox{and} \Psi _{x} C_{y} $ will be convenient later. The derivatives
 $\Psi _{y} C_{x} \spbox{and} \Psi _{x} C_{y} $ result in

\begin{equation} \label{ZEqnNum757716} \begin{array}{l} {\Psi _{y} C_{x} -\Psi
_{x} C_{y} =} \\ {\frac{\pi }{\xi } ^{2} \sum _{m=1}^{\infty } \sum
_{n=0}^{\infty } \sum _{r=0}^{\infty } \sum _{s=1}^{\infty }A_{m,n} B_{r,s}
[ms\int _{0}^{1} \int _{0}^{\xi }4\cos \left(g\pi y\right)\cos \left(m\pi
y\right)\cos \left(r\pi y\right) } \\ {\cdot \sin \left(h\pi \frac{x}{\xi}
\right)\cos \left(n\pi \frac{x}{\xi } \right)\cos \left(s\pi \frac{x}{\xi }
\right)dxdy} \\ {-nr\int _{0}^{1} \int _{0}^{\xi }4\cos \left(g\pi y\right)\sin
\left(m\pi y\right)\sin \left(r\pi y\right)} \\ {\sin \left(h\pi \frac{x}{\xi }
\right)\sin \left(n\pi \frac{x}{\xi } \right)\sin \left(s\pi \frac{x}{\xi }
\right)dxdy]} \end{array} \end{equation}

For the portion of the right hand side of equation \eqref{ZEqnNum183863}
represented by $\frac{1}{\xi } \Psi _{y} $ one obtains

\begin{equation} \label{ZEqnNum659914} \frac{\pi }{\xi } \sum _{m=1}^{\infty
} \sum _{n=0}^{\infty }A_{m,n} \cdot m\int _{0}^{1} \int _{0}^{\xi }4\cos
\left(g\pi y\right)\cos \left(m\pi y\right)\sin \left(h\pi \frac{x}{\xi }
\right)\sin \left(n\pi \frac{x}{\xi } \right) dxdy \end{equation}

The treatment of the terms $C_{x} \spbox{and} \frac{1}{\xi } $ in
\eqref{ZEqnNum183863}, is very similar to those of the right hand side of
equation \eqref{ZEqnNum626661}. Galerkin's Method should result in $C_{x} {\rm
\; and\; } \frac{1}{\xi } $ becoming

\begin{equation} \label{ZEqnNum392315} \frac{\pi }{\xi } \sum _{r=0}^{\infty
} \sum _{s=1}^{\infty } \left[B_{r,s} \cdot s\int _{0}^{1} \int _{0}^{\xi }4\cos
\left(g\pi y\right)\cos (r\pi y)\sin \left(h\pi \frac{x}{\xi } \right)\cos
\left(s\pi \frac{x}{\xi } \right)dxdy \right] \end{equation}s

and

\begin{equation} \label{ZEqnNum600143} \frac{1}{\xi } \int _{0}^{1} \int
_{0}^{\xi }4\cos \left(g\pi y\right)\sin \left(h\pi \frac{x}{\xi } \right)dxdy
\end{equation} 

respectively. In fact, the difference between integrals \eqref{ZEqnNum379567}
and \eqref{ZEqnNum600143} consists in $x\spbox{and}y$ switching positions.
So the result obtained from \eqref{ZEqnNum309765} can then be applied to
integral \eqref{ZEqnNum600143}. Therefore, one obtains

\begin{equation} \label{3.29)} \frac{1}{\xi } \int _{0}^{1} \int _{0}^{\xi }4\cos
\left(g\pi y\right)\sin \left(h\pi \frac{x}{\xi } \right)dxdy =\frac{4}{\pi }
W\left(h,g\right) \end{equation}

 The integral \eqref{ZEqnNum392315}, as stated before, is very similar to the
 integral \eqref{ZEqnNum851626}. In fact, the difference between integral
 \eqref{ZEqnNum392315} and integral \eqref{ZEqnNum851626} is that for
 \eqref{ZEqnNum392315} to be non-zero $r \text{ must\; equal\; } g$ instead of
 $s$equaling $h$, so it is as if the variables have been switched. One
 significant difference is that one must consider when $g=0$, resulting in

\begin{equation} \label{3.30)} \frac{\pi }{\xi } \sum _{r=0}^{\infty } \sum
_{s=1}^{\infty } \left[B_{r,s} \cdot s\int _{0}^{1} \int _{0}^{\xi }4\cos
\left(g\pi y\right)\cos (r\pi y)\sin \left(h\pi \frac{x}{\xi } \right)\cos
\left(s\pi \frac{x}{\xi } \right)dxdy \right] =\varepsilon \left(g\right)\cdot
\sum _{s=1}^{\infty }B_{g,s} \cdot s\cdot N\left(h,s\right) \end{equation}

The integral \eqref{ZEqnNum659914} is also very similar to that of
\eqref{ZEqnNum851626}, differing in that for \eqref{ZEqnNum659914} to be
non-zero, $m \text{ must\; equal\; } g$, resulting in

\begin{equation} \label{3.31)} \frac{\pi }{\xi } \sum _{n=0}^{\infty }A_{g,n}
\cdot g\int _{0}^{1}4\cos ^{2} \left(g\pi y\right)\int _{0}^{\xi } \sin
\left(h\pi \frac{x}{\xi } \right)\sin \left(n\pi \frac{x}{\xi } \right) dxdy
\end{equation}

By similar steps taken to evaluate \eqref{ZEqnNum929872}, one obtains

\begin{equation} \label{3.32)} \frac{\pi }{\xi } \sum _{n=0}^{\infty }A_{g,n}
\cdot g\int _{0}^{1}4\cos ^{2} \left(g\pi y\right)\int _{0}^{\xi } \sin
\left(h\pi \frac{x}{\xi } \right)\sin \left(n\pi \frac{x}{\xi } \right)
dxdy=\sum _{n=0}^{\infty }A_{g,n} \cdot g\cdot N\left(h,n\right) \end{equation} 



For \eqref{ZEqnNum757716} it will much easier to look at each integral
separately. First, look at the integral

\begin{equation*}
    \int _{0}^{1} \int _{0}^{\xi }4\cos \left(g\pi y\right)\cos \left(m\pi
    y\right)\cos \left(r\pi y\right)\sin \left(h\pi \frac{x}{\xi } \right)\cos
    \left(n\pi \frac{x}{\xi } \right)\cos \left(s\pi \frac{x}{\xi } \right)dxdy 
\end{equation*}

Using the relation

\begin{equation*} \begin{split}
    \sin \left(h\pi \frac{x}{\xi } \right)\cos \left(n\pi
    \frac{x}{\xi } \right)\cos \left(s\pi \frac{x}{\xi } \right) 
    & = \sin \left[\left(h+n+s\right)\pi \frac{x}{\xi } \right]
    + \sin \left[\left(h+n-s\right)\pi \frac{x}{\xi } \right] \\ 
    & + \sin \left[\left(h-n+s\right)\pi \frac{x}{\xi } \right]
    +\sin \left[\left(h-n-s\right)\pi \frac{x}{\xi } \right]  
\end{split} \end{equation*}

the integral becomes

\begin{equation*} \begin{split}
    \int _{0}^{1} 4 \cos \left(g \pi y\right) \cos \left(m \pi y\right)
    \cos \left(r \pi y\right) \int _{0}^{\xi } 
    &\sin \left[\left(h + n + s\right) \pi \frac{x}{\xi } \right]
    +\sin \left[\left(h+n-s\right)\pi \frac{x}{\xi } \right] \\ 
    & + \sin \left[\left(h - n + s\right)\pi \frac{x}{\xi } \right] 
    + \sin \left[\left(h - n - s\right)\pi \frac{x}{\xi } \right] dxdy 
\end{split} \end{equation*}

This then simplifies, using the relation \eqref{ZEqnNum821421}, to

\begin{equation*}
    \int _{0}^{1} 4\cos \left(g\pi y\right) 
    \cos \left(m\pi y\right)\cos \left(r \pi y\right)\cdot R dy 
\end{equation*}

Where

\begin{equation} \label{ZEqnNum244131} 
    \cfunc{R}
    {0}{h = n + s \csp h = n - s, \spbox{or} h = s - n}
    {\frac{\left(-1\right)^{h + n + s} - 1}{h + n + s} 
        + \frac{\left(-1\right)^{h + n - s} - 1}{h + n - s}
        + \frac{\left(-1\right)^{h - n + s} - 1}{h - n + s} 
        + \frac{\left(-1\right)^{h - n - s} - 1}{h - n - s}} 
    {\text{otherwise}}
\end{equation}

Then using the relation

\begin{equation*}
    \begin{array}{l} {\cos \left(g\pi y\right)\cos \left(m\pi y\right)\cos
    \left(r\pi y\right)=\cos \left[\left(g+m+r\right)\pi y\right]+\cos
    \left[\left(g+m-r\right)\pi y\right]} \\ {+\cos \left[\left(g-m+r\right)\pi
    y\right]+\cos \left[\left(g-m-r\right)\pi y\right]} \end{array} 
\end{equation*}

The integral now becomes 
\begin{equation*} \begin{split}
    \int_{0}^{1} 4(\cos \left[\left(g + m + r\right) \pi y\right]
    & + \cos \left[\left(g + m - r\right)\pi y\right] \\
    & + \cos \left[\left(g - m + r\right)\pi y\right]
    + \cos \left[\left(g - m - r\right)\pi y\right] ) Rdy 
\end{split} \end{equation*}
This integral is non-zero only when $g = m - r \csp 
g = r - m \text{, or when } g = m + r$. Using these facts to evaluate the previous
integral results in

\begin{equation*} \begin{split}
    \int _{0}^{1} 4 (\cos \left[\left(g + m + r\right)\pi y\right]
    & + \cos \left[\left(g + m - r\right) \pi y\right] \\
    & + \cos \left[\left(g - m + r\right) \pi y\right]
    + \cos \left[\left(g - m - r\right) \pi y\right] ) R dy = LR
\end{split} \end{equation*} 

where

\begin{equation} \label{ZEqnNum236446} L=\delta _{\left(m-r\right),g} +\delta
_{\left(r-m\right),g} +\delta _{\left(m+r\right),g} \end{equation}

The $\delta $ used in \eqref{ZEqnNum236446} is the Kronecker delta:

\begin{equation*}
    \cfunc{\delta _{i,j}}
    {0}{i \ne j}
    {1}{i = j}
\end{equation*}

Similarly the second portion of the integral \eqref{ZEqnNum757716} can be
evaluated in a very similar fashion resulting in

\begin{equation*}
    \int _{0}^{1} \int _{0}^{\xi} 4 \cos \left(g\pi y\right)
    \sin \left(m \pi y\right) \sin \left(r \pi y\right) 
    \sin \left(h \pi \frac{x}{\xi } \right)
    \sin \left(n \pi \frac{x}{\xi } \right)
    \sin \left(s\pi \frac{x}{\xi } \right) dxdy
    = FG
\end{equation*}

where
\begin{equation} \label{ZEqnNum668374} 
    \cfunc{G}
    {0}{\begin{array}{l} 
        h = n + s \csp h = n - s, \\ 
        \spbox{or} h = s - n
    \end{array}}
    {\begin{array}{ll} 
        \frac{\left(-1\right)^{h + n + s} - 1}{h + n + s}  
       + & \frac{\left(-1\right)^{h + n - s} - 1}{h + n - s} \\
       & - \frac{\left(-1\right)^{h - n + s} - 1}{h - n + s} 
        - \frac{\left(-1\right)^{h - n - s} - 1}{h - n - s}
    \end{array}}{\text{otherwise}} 
\end{equation}

and
\begin{equation} \label{3.36)} 
    F = \delta _{\left(m-r\right),g} 
    + \delta_{\left(r-m\right),g} - \delta _{\left(m + r\right), g} 
\end{equation}

Combining the two parts of \eqref{ZEqnNum757716} results in

\begin{equation} \label{3.37)} \begin{split} 
    & \frac{\pi }{\xi } ^{2} 
    \sum_{m=1}^{\infty } \sum _{n=0}^{\infty } \sum _{r=0}^{\infty } \sum _{s=1}^{\infty}
        A_{m,n} B_{r,s} m s \\
        & \int _{0}^{1} \int _{0}^{\xi} 4\cos \left(g \pi y\right) \cos \left(m\pi y\right)
            \cos \left(r\pi y\right) \cdot \sin \left( h\pi \frac{x}{\xi } \right)
            \cos \left(n\pi \frac{x}{\xi } \right)
            \cos \left(s\pi \frac{x}{\xi } \right) dxdy \\ 
        & - nr \int_{0}^{1} \int_{0}^{\xi} 4 \cos \left(g \pi y\right) \sin \left(m\pi y\right) 
            \sin \left(r \pi y\right) \sin \left(h\pi \frac{x}{\xi} \right) 
            \sin \left(n\pi \frac{x}{\xi } \right) \sin \left(s\pi \frac{x}{\xi} \right) dxdy \\
    & = \frac{\pi}{4} \sum _{m = 1}^{\infty} \sum_{n=0}^{\infty } \sum _{r=0}^{\infty } \sum _{s=1}^{\infty } 
        A_{m,n} B_{r,s} \left(msLR - nrFG\right)
\end{split} \end{equation} 

To finish equation \eqref{ZEqnNum183863}, the left hand side, the integral \eqref{ZEqnNum189325},
must be evaluated. It should be easy to see that the results are very similar to that of the
evaluation of the left hand side of equation \eqref{ZEqnNum626661}, that is the integral
\eqref{ZEqnNum923088}. And so, the evaluation of \eqref{ZEqnNum189325} results in

\begin{equation} \label{3.38)} \begin{array}{l} 
    b \pi ^{2} \sum _{r=0}^{\infty} \sum _{s=1}^{\infty } 
        \left[\left(r^{2} + \frac{s^{2} }{\xi ^{2} } \right) 
        B_{r,s} \int _{0}^{1} \int _{0}^{\xi } 
        4 \cos \left(g\pi y\right) \cos \left(r \pi y\right)
        \sin \left(s\pi \frac{x}{\xi} \right)
        \sin \left(h\pi \frac{x}{\xi} \right) dxdy \right] \\ 
    = \varepsilon (g)b\pi ^{2} B_{g,h} \left(g^{2}
        +\frac{h^{2} }{\xi ^{2} } \right)\xi
\end{array} \end{equation}

Putting all these results together one will arrive at

\begin{equation} \label{ZEqnNum704007} 
    \varepsilon \left(h\right)\cdot a \pi ^{2} A_{g,h} \left(g^{2} 
    + \frac{h^{2} }{\xi ^{2} } \right) \xi 
    = \sum _{r=0}^{\infty} B_{r,h} \cdot h \cdot N \left(g,r\right) 
    + \frac{4}{\pi} W \left(g, h\right)
\end{equation}

\begin{equation} \label{ZEqnNum765213} 
    \begin{array}{l} 
    \varepsilon \left(g\right) \cdot b \pi ^{2} B_{g,h} 
    \left(g^{2} + \frac{h^{2}}{\xi ^{2}}\right) \xi 
    = \frac{\pi }{4} 
    \sum _{m=1}^{\infty } \sum _{n=0}^{\infty } \sum_{r=0}^{\infty } \sum _{s=1}^{\infty}
    A_{m,n} B_{r,s} \left(msLR - nrFG\right) \\
    + \sum _{n=0}^{\infty} A_{g,n} \cdot g \cdot N \left(h, n\right) 
    + \varepsilon \left(g \right) \cdot \sum _{s=1}^{\infty} B_{g,s} 
    \cdot s\cdot N\left(h, s\right) + \frac{4}{\pi } W\left(g,h\right)
\end{array} \end{equation}

The notation used in equations \eqref{ZEqnNum704007} and \eqref{ZEqnNum765213}
have been defined below:

\begin{equation*}
    F = \delta _{\left(m - r\right), g} + \delta _{\left(r - m\right), g} - 
    \delta _{\left(m + r\right), g}
\end{equation*}

\begin{equation*}
    L=\delta _{\left(m - r\right), g} +\delta _{\left(r - m\right), g} + \delta
    _{\left(m + r\right), g} 
\end{equation*}

\begin{equation} \label{ZEqnNum668374} 
    \cfunc{G}
    {0}{\begin{array}{l} 
        h = n + s \csp h = n - s, \\ 
        \spbox{or} h = s - n
    \end{array}}
    {\begin{array}{ll} 
        \frac{\left(-1\right)^{h + n + s} - 1}{h + n + s}  
       + & \frac{\left(-1\right)^{h + n - s} - 1}{h + n - s} \\
       & - \frac{\left(-1\right)^{h - n + s} - 1}{h - n + s} 
        - \frac{\left(-1\right)^{h - n - s} - 1}{h - n - s}
    \end{array}}{\text{otherwise}} 
\end{equation}

\begin{equation*}
    \cfunc{R}
    {0}
    {h = n + s \csp h = n - s \spbox{or} h = s - n}
    {\frac{\left(-1\right)^{h + n + s} - 1}{h + n + s} +
    \frac{\left(-1\right)^{h + n - s} - 1}{h + n - s} + 
    \frac{\left(-1\right)^{h - n + s} - 1}{h - n + s} + 
    \frac{\left(-1\right)^{h - n - s} - 1}{h - n - s}}
    { \text{otherwise}}  
\end{equation*}

\begin{equation*}
    \cfunc{N\left(g,r\right)}
    {0}
    {r = g}
    {\frac{\left(-1\right)^{g + r} - 1}{g + r} + 
    \frac{\left(-1\right)^{g - r} - 1}{g - r}} 
    {otherwise} 
\end{equation*}

\begin{equation*}
    \ctfunc{W\left(g,h\right)}
    {0}
    {h \ne 0}
    {0}
    {g=0}
    {\frac{1 - \left(-1\right)^{g} }{g} }
    {h = 0}
\end{equation*}

\begin{equation*}
    \cfunc{\delta _{i,j} }
    {0}
    {i \ne j}
    {1}
    {i = j}
\end{equation*}

\begin{equation*}
    \cfunc{\varepsilon \left(g\right)}
    {2}
    {g = 0}
    {1}
    { \text{otherwise}}
\end{equation*}

For one to evaluate this numerically one must truncate the summations, since infinite sums cannot be
evaluated by computational methods. Therefore, an approximation to the true solution will be
obtained using the truncated versions of \eqref{ZEqnNum704007} and \eqref{ZEqnNum765213}.
