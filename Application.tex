A Multi-Variate Newton-Raphson's Method was used to solve the system of non-linear equations. An
initial guess is however required to begin Newton's Method. It was noticed that a guess of
$0\spbox{for} A\spbox{and} B$ could be made and still have convergence at a value of $b=0.1$.
However, for lower values of $b$, Newton's method would no longer converge using an initial guess of
$0$, therefore for $b<0.1$ an initial guess was obtained from the solution for $b=0.1$. This method
worked for values of $b\ge 6\times 10^{-3} $.

To determine the appropriate number of Fourier coefficients to be used, a base case for $a=0.263, \;
b=0.1,\spbox{and} \xi =2.0\spbox{with} i_{a} =i_{b} =j_{a} =j_{b} =15$ or 480 Fourier coefficients
was compared to the solution for which $i_{a} =i_{b} =j_{a} =j_{b} =20$ or 840 Fourier coefficients.
The $i$ and $j$ indices were again increased by five each for a total number of 1300 Fourier
coefficients and compared to the solution for the previous solution for 840 coefficients. This
process was repeated until little change was seen between the previous solution and the new
solution. Then the final number of terms would be used as the preferred number of terms. This
process resulted in a total number of 1860 Fourier coefficients being sufficient for proper
convergence. The increase from one set of coefficients to another resulted mostly in a change in the
upper 20\% of the problem domain.

Once the total number of Fourier coefficients has been determined, a simulation for $b=0.05$ was
performed at the specified number of Fourier coefficients. It was then desired to find out how low
the Newton's Method could go with respect to $b$. An iterative process was used for $b<0.05$,
whereby every time Newton's Method converged for a value of $b$ a new value of $b$ smaller than the
previous was used for attempting convergence. Using the initial guess obtained by the solution for
$b=0.1$, Newton's Method was able to converge down to a value of $b=6\times 10^{-3} $. Solutions for
values of $b<6\times 10^{-3} $ were attempted using the solution obtained by $b=6\times 10^{-3} $,
but results were difficult to obtain and were no longer consistent in terms of the expected results.
It would most likely be possible to obtain more consistent solutions for values of $b<6\times
10^{-3} $, if one were to use more Fourier coefficients, but due to time constraints this was not
tested. Therefore, only the results for $b=0.1$, 0.05, and $6\times 10^{-3} $are presented.

The results above were compared to that Henry's Method and SUTRA. Henry's Method requires one to
assume a value for the non-linear terms in equation \eqref{ZEqnNum765213}, i.e. $\frac{\pi }{4} \sum
_{m=1}^{\infty } \sum _{n=0}^{\infty } \sum _{r=0}^{\infty } \sum _{s=1}^{\infty }A_{m,n} B_{r,s}
\left(msLR-nrFG\right) =const$. This essentially linearizes the system of equations. Initially the
non-linear terms were assumed to be zero. After each iteration the non-linear terms are then updated
using the new estimated values of $A\spbox{and} B$. The resulting linear equations are such that the
equations for $B_{0,h} $ are linearly independent of $A_{g,h} \;  \text{and\; } B_{g>0,h} $. In
addition to the non-linear terms being treated as constants, Henry also treated the linear terms
involving only $A_{g,h} $ in equation \eqref{ZEqnNum765213} as a known value, i.e. $\sum
_{n=0}^{\infty }A_{g,n} \cdot g\cdot N\left(h,n\right) =const$. The value of these terms were
obtained using previous iteration of $B_{g>0,h} $ (initially this was assumed to be zero), and the
current iteration of $B_{0,h} $ substituted into equation \eqref{ZEqnNum704007}. Once the non-linear
and linear terms had been estimated the resulting system of linear equations involving $B_{g>0,h} $
were then solved. The newly obtained values for $B_{g>0,h} $ were then used to update the non-linear
terms. When a complete set of B's and A's were evaluated, a new set of C's and $\psi  \text{'s} $
are then evaluated. The process as described above was repeated until the error was less than some
epsilon. Where the error was defined to be 

\begin{equation} \label{5.1} 
    error = MAX\left(\left\| C_{new} -C_{old} \right\|_{\infty } ,\left\| 
    \psi _{new} -\psi _{old} \right\| _{\infty } \right)
\end{equation} 

and $epsilon=10^{-4} $. In this analysis the Fourier series was truncated in the fashion described
by S\'egol \cite{Segol};

\begin{center}
    $A_{1,0}$ through $A_{1,15}$
    $A_{2,1}$ through $A_{2,10}$
    $A_{3,0}$ through $A_{3,5}$
    $A_{4,1}$ through $A_{4,3}$
    $A_{5,0}$ through $A_{5,2}$
    $B_{0,1}$ through $B_{0,20}$
    $B_{1,1}$ through $B_{1,10}$
    $B_{2,1}$ through $B_{2,5}$
    $B_{3,1}$ through $B_{3,3}$
    $B_{4,1}$ through $B_{4,2}$
\end{center}

This truncation results in a total of 78 Fourier coefficients. 

The above method was used to solve for $b=0.1$ with an initial guess, for the nonlinear terms, of
zero. Once this solution was obtained the resulting values for the Fourier coefficients were used to
obtain an initial guess, for $b=0.09$. The same procedure was then used to obtain a solution for
$b=0.08$ using the solution for $b=0.09$. This was repeated until a solution for $b=0.05$ was
obtained.

For SUTRA a scheme that would model Henry's Solution for $a=0.263 \spbox{and} b=0.1$, 0.05, and
$6\times 10^{-3} $ needed to be established, before comparing results. In previous papers such as
Voss and Souza \cite{Voss}, comparisons using Henry's problem and SUTRA have never seemed to use
consistent boundary conditions. That is, Henry's problem uses a fixed boundary condition for
concentration on the seaward side, but many comparison of SUTRA to Henry's solution did not hold
concentration constant on the seaward boundary. This results in both a difference in concentrations
for the upper 20\% of the solution domain and a differing position for the seawater lens. Thus,
using such a comparison will not result in similar solution, and should not be used as a comparison,
unless a comparison of shape is wanted for the bottom 80\%.  However, in the SUTRA code one can
change the boundary condition, so that the seawater side has a constant concentration, and so this
feature of SUTRA will be used in the comparison that follows.

Along with boundary condition, one must also match dispersion models. Henry used a simplified model
of dispersion, where the total dispersion is not dependent on velocity, SUTRA, on the other hand,
uses a total dispersion that is the sum of the velocity dependent dispersion and molecular
diffusivity multiplied by porosity. To be able to compare a SUTRA model to Henry's model one would
set the dispersivity to zero and use an artificially large molecular diffusivity. 

With both the boundary conditions and dispersion models aligned correctly, one can now proceed to
matching parameter distributions. Henry's problem uses parameters such as $a \spbox{and} b$, whereas
SUTRA uses parameters such as freshwater recharge ($Q$), porosity ($\varepsilon $), hydraulic
conductivity ($k$), seawater density ($\rho _{s}$), freshwater density ($\rho _{0}$), salt
concentrations ($C_{s}$), and the dispersion coefficient ($D$), and so it is necessary to determine
the parameter values that will represent the Henry's non-dimensional parameters $a \spbox{and} b$.
The parameter values to be used, in the SUTRA model, will be similar to that of Voss and Souza
\cite{Voss}, and are:

$\varepsilon =0.35$

$C_{s} =0.0357 \; \left[\frac{ \text{kg(dissolved solids)} }{
\text{kg(seawater)} } \right]$

$\rho _{s} =1024.99 \text{ kg/m} ^{3} $ 

$\pderiv{\rho}{C} = 700 \; \left[\frac{ \text{kg} \left( \text{seawater}
\right)^{2} }{\left( \text{kg} \left( \text{dissolved solids} \right) \text{ m}
^{3} \right)} \right]$

$\rho _{0} =1000 \text{ kg/m} ^{3} $ 

$Q=6.6\times 10^{-2} \text{ kg/s} $

$k= 1.024698 \times 10^{9} $ (Voss and Souza used $k= 1.020408 \times 10^{9} $)

$\left|g\right| = 9.8 \text{ m/s} ^{2} $

$\alpha _{L} = \alpha _{T} =0 $ 

$B=1.0 \text{ m} $ 

$D_{m} =18.85714\times 10^{-6} \text{ m/s} ^{2} \spbox{for} b = 0.1$

$D_{m} = 9.428571 \times 10^{-6} \text{ m/s} ^{2} \spbox{for} b=0.05$

$D_{m} = 1.131429 \times 10^{-6} \text{ m/s} ^{2} \spbox{for} b = 6\times 10^{-3} $

$C_{0} =0$ (concentration of freshwater)

The hydraulic conductivity differs from Voss and Souza, because the values used by Voss and Souza
($k=1.020408 \times 10^{-2} \text{ kg/s} $) did not seem to result in the appropriate values of $a
\spbox{and} b$, for all other given parameter values. As one can see three different values of
molecular diffusivity were used. These three different values were used to match the three different
values of$b$that were used in Henry's problem. No other parameter values were changed during the
simulation, since changing molecular diffusivity is sufficient to match the different values of $b$.
Total simulated time and total number of nodes for the SUTRA simulation were the same as that used
by Voss and Souza \cite{Voss}, 100 minutes and a total 231 nodes.
