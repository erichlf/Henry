Using Newton's Method to evaluate Henry's problem results in the ability to use a greater number of
Fourier coefficients than that of Henry's Method. This allows for an increase in the accuracy of the
solution. The increase in solution accuracy further allows for simulations of narrower zones of
dispersion (small values of $b$). The lower values of $b$ correspond to lower amounts of dispersion,
and therefore address the concerns stated by Voss and Souza \cite{Voss}, in which they say,
``because of the unrealistic large amount of dispersion introduced in the solution by the constant
total dispersion coefficient, this test does not check whether a model is consistent or whether it
accurately represents density driven flows, nor does it check whether a model can represent field
situations with relatively narrow transition zones.'' 

It appears that a further increase in the total number of Fourier coefficients may be necessary to
address the issue of instabilities created when evaluating Henry's Problem for low values of b. Due
to time constraints the number of Fourier coefficients for this analysis was limited to 1860
coefficients. It is believed that the number of Fourier coefficients to quell the instabilities may
be in excess of 5100 coefficients. The majority of time needed for each iteration was in the
evaluation of the Jacobian matrix $D\Phi $. A quicker algorithm for calculating the total derivative
function may be able to be developed, and therefore it may be possible to evaluate Henry's Problem
using a greater number of Fourier coefficients in a more timely fashion.

In addition to developing a more efficient algorithm for the total derivative function, it might be
necessary to use a more robust numerical method than that of Newton's Method. If a more robust
numerical method is used it may be able to address the instabilities in the upper 20\% of the
solution domain, without a further increase in the number of Fourier coefficients used in the
evaluation of Henry's Problem. It is clear that using Newton's Method over that of Henry's Method
increases the stability of Henry's Problem. This increase in stability resulted in the ability to
use a larger number of Fourier coefficients, and therefore an increase in the accuracy of the
solution. The increase in solution accuracy then resulted in the ability to evaluate Henry's Problem
for values of $b<0.05$. Using an even more robust numerical method may have similar results.
