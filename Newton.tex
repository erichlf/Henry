To improve the results obtained by Henry's Analytic solution a Newton-Raphson Method was used.
Originally Henry used a method simple method of linearization, which will be discussed in depth
later. The method originally used by Henry will be discussed in depth later. For Newton's Method to
make sense, in the contexts of Henry's Problem, it is necessary to define a vector $X$ made up of
vectors $A \spbox{and} B$, therefore let

\begin{equation} \label{4.1)} 
    X = \begin{bmatrix}
    A_{1,0} \\ 
    A_{1,1} \\ 
    \vdots \\ 
    A_{i_{a} ,j_{a} } \\ 
    B_{0,1} \\ 
    B_{0,2} \\ 
    \vdots \\
    B_{i_{b} ,j_{b} } \end{bmatrix} \; 
    i_{a} \csp i_{b} \csp j_{a} \csp j_{b} \in \mathbb{Z}^{+} 
\end{equation}

It is also necessary to define a vector function based upon
\eqref{ZEqnNum704007} and \eqref{ZEqnNum765213}, where

\begin{equation} \label{4.2)} 
    \phi _{g,h} \left(X\right) = \varepsilon \left(h \right) a \pi ^{2} 
    \cdot A_{g,h} \cdot \left(g^{2} + \frac{h^{2} }{\xi ^{2}} \right)
    \xi - \sum _{r=0}^{i_{b} }B_{r,h} \cdot h \cdot N\left(g,r\right)
    - \frac{4}{\pi } W\left(g,h\right) 
\end{equation} 
and
\begin{equation} \label{4.3)} \begin{array}{l} 
    \gamma _{g,h} \left(X\right) = \varepsilon \left(g \right) b \pi ^{2} 
    \cdot B_{g,h} \cdot \left(g^{2} + \frac{h^{2}}{\xi ^{2}} \right) \xi 
    - \sum _{n=0}^{i_{a} }A_{g,n} \cdot g \cdot N\left(h,n\right) \\ 
    - \varepsilon \left(g\right) \sum_{s=1}^{i_{b}} 
    B_{g,s} \cdot s \cdot N\left(h,s\right) - \frac{\pi}{4} 
    \sum_{m=1}^{i_{a}} \sum _{n=0}^{j_{a} } \sum _{r=0}^{i_{b}} \sum _{s=1}^{j_{b}} 
    A_{m,n} B_{r,s} \left(msLR - nrFG\right) - \frac{4}{\pi} W\left(h,g\right)
\end{array} \end{equation}

Newton's Method however, takes only one function and therefore these two vector
functions must be combined to make one new vector function. We solve it as a
system and combine both of the above functions in one vector:

\begin{equation} \label{ZEqnNum407215} \Phi
\left(X\right)=\left[\begin{array}{c} {\phi \left(X\right)} \\ {\gamma
\left(X\right)} \end{array} \right] \end{equation}

Let $X^{(n+1)} $be the value of $X$ at the $n+1^{th} $ iteration of the
Multi-Variate Newton-Raphson's Method. We then can write Newton's Method as

\begin{equation} \label{4.5)} X^{\left(n+1\right)} =X^{\left(n\right)} -D\Phi
^{-1} \left(X^{\left(n\right)} \right)\cdot \Phi \left(X^{\left(n\right)}
\right), \text{ \; \; } n\in {\mathbb N} \end{equation}

where $D\Phi \left(X^{\left(n\right)} \right)$ is the Jacobian matrix evaluated
at $X^{\left(n\right)} \spbox{and} D\Phi ^{-1} \left(X^{\left(n\right)} \right)$ is
the inverse to $D\Phi \left(X^{\left(n\right)} \right)$.

For simplicity let $\Phi ^{\left(n\right)} \spbox{denote} \Phi
\left(X^{\left(n\right)} \right)$, $\phi ^{\left(n\right)} {\rm \; denote\;
} \phi \left(X^{\left(n\right)} \right)$, $\gamma ^{\left(n\right)} {\rm \;
denote\; } \gamma \left(X^{\left(n\right)} \right)$, $D\Phi ^{\left(n\right)}
\spbox{denote}D\Phi \left(X^{\left(n\right)} \right)$, and $D\Phi
_{\left(n\right)}^{-1} \spbox{denote}D\Phi ^{-1} \left(X^{\left(n\right)}
\right)$, therefore one arrives at

\begin{equation} \label{Jacobian)} 
    D\Phi ^{\left(n\right)} =\left[\begin{array}{cccc} 
    {\pard{\phi_{1,0}^{(n)}}{X_{1}^{(n)}} } 
    & {\pard{\phi_{1,0}^{(n)}}{X_{2}^{(n)}} } 
    & {\cdots } 
    & {\pard{\phi_{1,0}^{(n)}}{X_{i_{a} \left(j_{a} +1\right)+\left(i_{b} + 1\right)j_{b} }^{(n)}} } \\ 
    
    {\pard{\phi_{1,1}^{(n)}}{X_{1}^{(n)}} } 
    & {\pard{\phi_{1,1}^{(n)}}{X_{2}^{(n)}} } 
    & {\cdots } 
    & {\pard{\phi_{1,1}^{(n)}}{X_{i_{a} \left(j_{a} +1\right)+\left(i_{b} +1\right)j_{b} }^{(n)}} } \\ 
    
    {\vdots } & {\vdots } & {\ddots } & {\vdots } \\ 
    
    {\pard{\phi_{i_{a} ,j_{a} }^{(n)}}{X_{1}^{(n)}} } 
    & {\pard{\phi_{i_{a},j_{a} }^{(n)}}{X_{2}^{(n)}} } 
    & {\cdots } 
    & {\pard{\phi_{i_{a} ,j_{a}}^{(n)}}{X_{i_{a} \left(j_{a} +1\right)+\left(i_{b} +1\right)j_{b} }^{(n)}} } \\

    {\pard{\gamma_{0,1}^{(n)}}{X_{1}^{(n)}}} 
    & {\pard{\gamma_{0,1}^{(n)}}{X_{2}^{(n)}} } 
    & {\cdots } 
    &{\pard{\gamma_{0,1}^{(n)}}{X_{i_{a} \left(j_{a} +1\right)+\left(i_{b}+1\right)j_{b} }^{(n)}} } \\

    {\pard{\gamma_{0,2}^{(n)}}{X_{1}^{(n)}} } 
    & {\pard{\gamma_{0,2}^{(n)}}{X_{2}^{(n)} }} 
    & {\cdots } 
    & {\pard{\gamma_{0,2}^{(n)}}{X_{i_{a} \left(j_{a}+1\right)+\left(i_{b} +1\right)j_{b} }^{\left(n\right)} }} \\ 
    
    {\vdots } & {\vdots } & {\ddots } & {\vdots } \\

    {\pard{\gamma_{i_{b} ,j_{b}}^{(n)}}{X_{1}^{(n)}} } 
    & {\pard{\gamma_{i_{b} ,j_{b} }^{(n)}}{X_{2}^{(n)}} }
    & {\cdots } 
    & {\pard{\gamma_{i_{b} ,j_{b} }^{(n)}}{X_{i_{a} \left(j_{a}+1\right)+\left(i_{b} +1\right)j_{b} }^{(n)}} } 
\end{array} \right] \end{equation}

Where $i_{a} $ is the number of rows of $A$, $j_{a} $ is the number of columns
of $A$, $i_{b} $ is the number of rows of $B$, and $j_{b} $ is the number of
columns of $B$, therefore

\begin{equation} \label{ZEqnNum134946} \begin{array}{l} {\left|\phi
\right|=\left|A\right|=i_{a} \left(j_{a} +1\right)} \\ {\left|\gamma
\right|=\left|B\right|=j_{b} \left(i_{b} +1\right)} \end{array} \end{equation}

To understand how to write the algorithm for this specific use of Newton's
Method, one must first understand the structure of $\Phi \spbox{and} D\Phi $. The
vector function $\Phi $ has two parts as was shown in \eqref{ZEqnNum407215}. The
first portion of the vector function \eqref{ZEqnNum407215} is that portion which
represents the equations for the A's. This is the portion of $\Phi $ given by
$\phi (X)$. The second portion of the vector function \eqref{ZEqnNum407215} is
that portion for which the equations for the B's are represented, and is the
portion of $\Phi $ given by $\gamma (X)$.

The Jacobian matrix $D\Phi \left(X\right)$ has a similar split along its
columns. Where the first portion is given by the derivatives of the
function$\phi \left(X\right)$, and the second portion is given by the
derivatives of the function$\gamma \left(X\right)$. But, now there is an
additional split along the rows; portion one being the derivatives of the
vector function $\Phi \left(X\right)$ with respect to the A's, and portion two
being the derivatives of the vector function $\Phi \left(X\right)$ with respect
to the B's. Therefore, the matrix $D\Phi $ is actually split into four
quadrants. The matrix therefore can be broken down into its component sections
as follows

\begin{equation*}
    D \Phi = \begin{bmatrix} 
    I & II \\ 
    III & IV \end{bmatrix}
\end{equation*} 

Where section I, II, III, and IV represent $\pard{\phi}{A}$, $\pard{\phi}{B}$,
$\pard{\gamma}{A}$, and $\pard{\gamma}{A}$ respectively, and can be written

\begin{equation} \label{4.8)} 
    D \Phi = \begin{bmatrix}
    \pard{\phi}{A} & \pard{\phi}{B} \\ 
    \pard{\gamma}{A} & \pard{\gamma}{B} \end{bmatrix}
\end{equation}

Where

\begin{equation*}
    dim\left(I\right)=i_{a} \left(j_{a} +1\right)\times i_{a}\left(j_{a} +1\right)
\end{equation*} 

\begin{equation*}
    dim\left(II\right)=i_{a} \left(j_{a} +1\right)\times j_{b} \left(i_{b} +1\right) 
\end{equation*} 

\begin{equation*}
    dim\left(III\right)=j_{b} \left(i_{b} +1\right)\times i_{a} \left(j_{a} +1\right) 
\end{equation*} 

\begin{equation*}
    dim\left(IV\right)=j_{b} \left(i_{b} +1\right)\times j_{b} \left(i_{b} +1\right) 
\end{equation*} 

and therefore

\begin{equation} \label{4.9)} 
    dim(D\Phi )=\left[i_{a} \left(j_{a} + 1\right)
    + \left(i_{b} +1\right)j_{b} \right] 
    \times \left[i_{a} \left(j_{a} + 1\right)
    + \left(i_{b} + 1\right) j_{b} \right]
\end{equation}

The size of each quadrant is dependent on the size of the vectors$\phi $
and$\gamma $, defined in \eqref{ZEqnNum134946}, which in turn is determined by
the size of the vectors $A\spbox{and}B$.

The number of rows depends on which function ($\phi \spbox{or} \gamma $) is
being used, and the number of columns depends on whether the quadrant takes the
derivative with respect to $A\spbox{or}B$. To encode this matrix, one must
translate a loop counter into the indices for $A\spbox{and}B$. Since
quadrants I and III are derivatives with respect to $A$, they have one pattern,
and quadrants II and IV are derivatives with respect to $B$, they have another
pattern.

The matrix $D\Phi $ is a square matrix; therefore a main diagonal is defined by
the derivatives of $\phi $and $\gamma $, with respect to$A_{g,h} $ and the
$B_{g,h} $ terms, respectively. For these terms, one only needs to determine a
pattern that translates the row number in the matrix $D\Phi \spbox{into} g\spbox{and} h$,
and then place the diagonal derivatives (e.g. $\pard{\phi_{g,h}}{A_{g,h}} $ and
$\pard{\gamma_{g,h}}{B_{g,h}} $) into the main diagonal of $D\Phi $. For the
rest of the matrix $D\Phi $ the pattern will begiven by the column. For
quadrants I and II the pattern is given by the function $\phi $, and for
quadrants III and IV the pattern is given by that of the function $\gamma $. If
the column counter is less than or equal to the total number of A's then the
pattern for $A$ is used. If the counter for the columns is greater than that of
the total number of $A$'s the pattern for $B$ is used. If the counter for the
rows is less than or equal to the total number of $\phi $ functions then the
equation for the function $\phi $ is used, while if the counter for the rows is
greater than the total number of $\phi $ functions then the equations for the
function $\gamma $ are used.

Let $i$ be the counter for the rows, then the pattern that translates the row
number $i\spbox{into} g\spbox{and} h$ can be discovered by writing out a table.

\textbf{Table 1}

\begin{center} \begin{tabular}{|p{1.0in}|p{0.2in}|p{0.2in}|} \hline 
    $i$ & $g$ & $h$ \\ \hline 
    1 & 1 & 0 \\ \hline 
    2 & 1 & 1 \\ \hline 
    3 & 1 & 2 \\ \hline 
    $\vdots $ & $\vdots $ & $\vdots $ \\ \hline 
    $j_{a} +1$ & 1 & $j_{a} $ \\ \hline 
    $j_{a} +2$ & 2 & 0 \\ \hline 
    $\left(j_{a} +2\right)+1$ & 2 & 1 \\ \hline 
    $\vdots $ & $\vdots $ & $\vdots $ \\ \hline 
    $\left(j_{a} +2\right)+j_{a} $ & 2 & $j_{a} $ \\ \hline 
    $2j_{a} +3$ & 3 & 0 \\ \hline 
    $\left(2j_{a} +3\right)+1$ & 3 & 1 \\ \hline 
    $\vdots $ & $\vdots $ & $\vdots $ \\ \hline 
    $\left(\left(i_{a} -1\right)j_{a} +i_{a} \right)+j_{a} $ & $i_{a} $ & $j_{a} $
    \\ \hline
    $i_{a} \left(j_{a} +1\right)+1$ & 0 & 1 \\ \hline 
    $i_{a} \left(j_{a} +1\right)+2$ & 0 & 2 \\ \hline 
    $\vdots $ & $\vdots $ & $\vdots $ \\ \hline 
    $i_{a} \left(j_{a} +1\right)+j_{b} $ & 0 & $j_{b} $ \\ \hline 
    $i_{a} \left(j_{a} +1\right)+j_{b} +1$ & 1 & 1 \\ \hline 
    $i_{a} \left(j_{a} +1\right)+j_{b} +2$ & 1 & 2 \\ \hline 
    $\vdots $ & $\vdots $ & $\vdots $ \\ \hline 
    $i_{a} \left(j_{a} +1\right)+j_{b} +j_{b} $ & 1 & $j_{b} $ \\ \hline 
    $i_{a} \left(j_{a} +1\right)+2j_{b} +1$ & 2 & 1 \\ \hline 
    $\vdots $ & $\vdots $ & $\vdots $ \\ \hline 
    $i_{a} \left(j_{a} +1\right)+j_{b} \left(i_{b} +1\right)$ & $i_{b} $ & $j_{b} $
    \\ \hline
\end{tabular} \end{center}

Here one can see that when $i\le i_{a} \left(j_{a} +1\right)$ there is one
pattern, and when $i>i_{a} \left(j_{a} +1\right)$ there is another pattern.
Therefore, when $i\le i_{a} \left(j_{a} +1\right)$ one arrives at the
relationships 

\begin{equation*}
    \left(p - 1\right) j_{a} + p \le i \le p(j_{a} + 1) \to g = p 
    \quad p \in \mathbb{N} 
\end{equation*}

\begin{equation*}
    \cfunc{h}
    {j_{a}}{\spbox{if} mod(i,j_{a} + 1) = 0} 
    {mod(i,j_{a} +1) - 1}{\text{otherwise}} 
\end{equation*}

and when $i>i_{a} \left(j_{a} +1\right)$ one arrives at the similar relations
\begin{equation*}
    p \cdot j_{b} + 1 \le i - i_{a} \left(j_{a} + 1\right) \le \left(p + 1\right) j_{b} \to g = p
    \quad p \in \mathbb{Z}^{+}
\end{equation*}

\begin{equation*}
    \cfunc{h}
    {j_{b}}{\spbox{if} mod(i - i_{a} \left(j_{a} + 1\right), j_{b})=0}
    {mod(i - i_{a} \left(j_{a} + 1\right), j_{b})}{\text{otherwise}}
\end{equation*}

Now that the pattern has been established one can use this pattern inside a set
of ``Loops'' to produce $D\Phi ^{\left(n\right)} \spbox{and} \Phi ^{\left(n\right)}
$.

To initialize $D\Phi $ one needs to translate the pattern as described in Table
1 into an easy way of taking those derivatives. The derivative will depend on
which variable and which function one is working with. This will be determined
by both the row and column of the matrix. Where the row indicates the function
to be used and the column determines the which variables derivative to take. In
general one has:

\textbf{Section I}

\begin{equation} \label{4.10)} 
    \cfunc{\pderiv{\phi_{g,h}}{A_{m,n}}}
    {\varepsilon \left(h\right)a\pi ^{2} \cdot 
    \left(g^{2} +\frac{h^{2} }{\xi ^{2} } \right) \xi }{ \text{if } m = g \spbox{and} n = h } 
    {0}{\text{otherwise}}
\end{equation}

\textbf{Section II}

\begin{equation} \label{4.11)} 
    \cfunc{\pderiv{\phi_{g,h}}{B_{r,s}}}
    {-h \cdot N\left(g,r\right)}{\text{if } r \ne g \spbox{and} s = h}
    {0}{\text{otherwise}}
\end{equation}

\textbf{Section III}

\begin{equation} \label{4.12)} 
    \cfunc{\pderiv{\gamma_{g,h}}{A_{m,n}}}
    {-g \cdot N\left(h,n\right) 
        - \frac{\pi}{4} \sum _{r=0}^{i_{b} } \sum _{s=1}^{j_{b}} B_{r,s} 
        \cdot \left(gsLR - nrFG\right)} 
    {\text{if } m = g \spbox{and} n \ne h}
    {-\frac{\pi }{4} \sum _{r=0}^{i_{b} } \sum _{s=1}^{j_{b}} B_{r,s} 
        \cdot \left(msLR - nrFG\right)} 
    {\text{otherwise}}
\end{equation}

\textbf{Section IV}

\begin{equation} \label{4.13)} 
    \pderiv{\gamma_{g,h}}{B_{r,s}} = 
    \begin{cases} 
    \varepsilon \left(g\right)b\pi ^{2} 
            \cdot \left(g^{2} + \frac{h^{2} }{\xi ^{2} } \right) \xi 
            - \frac{\pi }{4} \sum _{m=1}^{i_{a} } \sum _{n=0}^{j_{a} } A_{m,n} 
            \cdot \left(mhLR-ngFG\right)   
    & \text{if } r = g \spbox{and} s = h \\ 
     -s \cdot N\left(h, s\right) - \frac{\pi }{4} \sum _{m=1}^{i_{a} } \sum _{n=0}^{j_{a} }A_{m,n}
            \cdot \left(msLR - ngFG\right)
    & \text{if } r = g \spbox{and} s \ne h \\
     -\frac{\pi }{4} \sum _{m=1}^{i_{a} } \sum _{n=0}^{j_{a}} A_{m,n} 
            \cdot \left(mhLR - nrFG\right)
    & \text{if } r \ne g \spbox{and} s = h \\ 
     -\frac{\pi }{4} \sum _{m=1}^{i_{a} } \sum _{n=0}^{j_{a}} A_{m,n} 
            \cdot \left(msLR - nrFG\right)
    \text{otherwise} \end{cases}
\end{equation} 



